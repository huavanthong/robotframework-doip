% --------------------------------------------------------------------------------------------------------------
%
% Copyright 2020-2023 Robert Bosch GmbH

% Licensed under the Apache License, Version 2.0 (the "License");
% you may not use this file except in compliance with the License.
% You may obtain a copy of the License at

% http://www.apache.org/licenses/LICENSE-2.0

% Unless required by applicable law or agreed to in writing, software
% distributed under the License is distributed on an "AS IS" BASIS,
% WITHOUT WARRANTIES OR CONDITIONS OF ANY KIND, either express or implied.
% See the License for the specific language governing permissions and
% limitations under the License.
%
% --------------------------------------------------------------------------------------------------------------

% Introduction

\section{Overview}

\pkg\ is a Robot Framework library specifically designed for interacting with Electronic Control Units (ECUs) using 
the Diagnostics over Internet Protocol (DoIP).

At its core, DoIP serves as a communication bridge between external diagnostic tools and a vehicle's ECUs. This library, RobotFrameworkDoIP, 
provides a set of keywords that enable users to perform diagnostic operations and engage with ECUs, facilitating automated testing processes 
and interaction with vehicles through the DoIP protocol.

The \pkg\ sources can be found in repository \repo:
\href{https://github.com/test-fullautomation/robotframework-doip}{DoIP}


\section{Abbreviations}

\begin{table}[htbp]
  \centering
  \caption{Abbreviation}
  \label{tab:abbreviations}
  \begin{tabular}{|m{3cm}|m{7cm}|}
    \hline
    \textbf{Abbreviation / Acronym} & \textbf{Description} \\
    \hline
    ARP & Address Resolution Protocol \\
    \hline
    DHCP & Diagnostic Host Configuration Protocol \\
    \hline
    EID & Entity identifier \\
    \hline
    GID & Group identifier \\
    \hline
    ICMP & Internet Control Message Protocol \\
    \hline
    IP & Internet Protocol \\
    \hline
    IPv4 & Internet Protocol version 4 \\
    \hline
    IPv6 & Internet Protocol version 6 \\
    \hline
    TCP & Transmission Control Protocol \\
    \hline
    TCP/IP & A family of communication protocols used in computer networks \\
    \hline
    VIN & Vehicle Identification Number \\
    \hline
    UDP & User Datagram Protocol \\
    \hline
  \end{tabular}
\end{table}



\section{Terms and definitions}

\subsection{Diagnostic Power Mode}
The vehicle internal power supply status affects the diagnostic capabilities of all ECUs on the in-vehicle network 
and identifies the status of all ECUs in all gateway subnetworks that allow diagnostic communication.
Its purpose is to provide external test equipment with whether it can be performed on the connected vehicle.

\subsection{Transport protocol}
Transport protocols reside on top of the IP protocol. Transport protocols run over the best-effort IP layer to provide
a mechanism for applications to communicate with each other without directly interacting with the IP layer.

The important thing here is that DoIP does not represent a diagnostic protocol according to ISO 13400 but rather 
an expanded transport protocol. This means that the transmission of diagnostic packets is defined in DoIP, but the 
contained diagnostic services continue to be specified and described by diagnostic protocols such as KWP2000 and UDS.

\subsection{Diagnostics tester} 
A diagnostic tester / diagnostic client enables the sending of diagnostic requests. Testers can take the form of 
external devices, such as in repair shops, or on-board testers in the vehicle. 
The receiving ECU must process the diagnostic requests and return an associated diagnostic response to the tester.

\subsection{Diagnostics gateway}
The gateway assumes the role of the intermediary. 
Requests of the tester are forwarded to internal networks so that a desired ECU can receive and process them. As 
soon as a response from the requested ECU is available, the gateway routes this back to the tester.

\subsection{Logical addresses} 
A diagnostic gateway always requires two pieces of information in order to forward diagnostic requests and responses. 
  \begin{itemize}
  \item First, it requires a logical address that uniquely identifies the ECU to be diagnosed in the vehicle, equivalent 
  to the ecu logical address parameter.

  \item Second, the gateway must know which messages on the respective bus system or network will be used to send diagnostic
   requests and to receive diagnostic responses, equivalent to client logical addressp parameter.
  \end{itemize}
Both pieces of information must be available for an ECU for it to be accessible via the gateway.
