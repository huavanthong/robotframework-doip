% Copyright 2020-2023 Robert Bosch GmbH

% Licensed under the Apache License, Version 2.0 (the "License");
% you may not use this file except in compliance with the License.
% You may obtain a copy of the License at

% http://www.apache.org/licenses/LICENSE-2.0

% Unless required by applicable law or agreed to in writing, software
% distributed under the License is distributed on an "AS IS" BASIS,
% WITHOUT WARRANTIES OR CONDITIONS OF ANY KIND, either express or implied.
% See the License for the specific language governing permissions and
% limitations under the License.

% --------------------------------------------------------------------------------------------------------------

\label{ecusimulator}

This chapter provides a detailed explanation of the utilization of the ECU simulator through DoIP base on doipclient library. It serves for development or testing scenarios where a physical device is not available.
\\The ECU simulator is designed to receive messages and respond accordingly to the following types of messages:

\begin{itemize}
   \item Alive Check Request
   \item Diagnostic Power Mode Request
   \item Doip Entity Status Request
   \item Routing Activation Request
   \item Vehicle Identification Request
\end{itemize}

\section{Initialize}

This function sets up an instance of an ECU, initializes its attributes with default values, and includes placeholders for various properties that can be customized based on specific requirements.

\begin{pythoncode}
   def __init__(self, ecu_type, ip_address, tcp_port, udp_port):
      # Initialize ECU attributes with default values
      self.ecu_type = ecu_type
      self.ip_address = ip_address
      self.tcp_port = tcp_port
      self.udp_port = udp_port
      self.tcp_socket = None
      self.udp_socket = None
      # Set default values for various ECU properties
      # These values might be placeholders and can be updated based on your actual requirements
      self._ecu_logical_address = 3584
      self._client_logical_address = 3584
      self._logical_address = 55
      self._response_code = doip_message.RoutingActivationResponse.ResponseCode.Success
      self._diagnostic_power_mode = doip_message.DiagnosticPowerModeResponse.DiagnosticPowerMode.Ready
      self._node_type = 1
      self._max_concurrent_sockets = 16
      self._currently_open_sockets = 1
      self._max_data_size = None
      self._vin = '19676527011956855057'
      self._eid = b'11111'
      self._gid = b'222222'
      self._further_action_required = doip_message.VehicleIdentificationResponse.FurtherActionCodes.NoFurtherActionRequired
      self._vin_sync_status = doip_message.VehicleIdentificationResponse.SynchronizationStatusCodes.Synchronized
\end{pythoncode}

\newpage

\section{Start}

This method is responsible for initializing and setting up TCP and UDP sockets, binding them to specific IP addresses and ports, and then starting separate threads to handle the communication on these sockets concurrently.

\begin{pythoncode}
   def start(self):
      # Create TCP socket
      self.tcp_socket = socket.socket(socket.AF_INET, socket.SOCK_STREAM)
      self.tcp_socket.bind((self.ip_address, self.tcp_port))
      self.tcp_socket.listen(5)

      # Create UDP socket
      self.udp_socket = socket.socket(socket.AF_INET, socket.SOCK_DGRAM)
      self.udp_socket.bind((self.ip_address, self.udp_port))

      # Start listening on separate threads
      tcp_thread = threading.Thread(target=self.listen_tcp)
      udp_thread = threading.Thread(target=self.listen_udp)

      tcp_thread.start()
      udp_thread.start()
\end{pythoncode}

\textbf{Explanation:}
\begin{enumerate}
   \item TCP Socket Setup
      \begin{itemize}
         \item A TCP socket is created using the socket module with the \plog{socket.AF_INET} family (IPv4) and \plog{socket.SOCK_STREAM} type (TCP).
         \item The TCP socket is bound to the specified IP address \plog{self.ip_address} and TCP port \plog{self.tcp_port}.
         \item The TCP socket is set to listen for incoming connections with a backlog of 5 connections.
      \end{itemize}
   \item UDP Socket Setup
      \begin{itemize}
         \item A UDP socket is created using the same socket module with the \plog{socket.AF_INET} family (IPv4) and \plog{socket.SOCK_DGRAM type} (UDP).
         \item The UDP socket is bound to the specified IP address \plog{self.ip_address} and UDP port \plog{self.udp_port}.
      \end{itemize}
   \item Thread Creation
   \begin{itemize}
      \item Two separate threads \plog{tcp_thread and udp_thread} are created using the threading module.
      \item The target parameter of each thread is set to point to specific methods \plog{self.listen_tcp} and \plog{self.listen_udp}, suggesting that these methods likely contain the logic for handling TCP and UDP communication.
   \end{itemize}
   \item Thread Start
   \begin{itemize}
      \item Both threads are started concurrently using the start method, allowing the ECU to handle TCP and UDP communication simultaneously.
   \end{itemize}
\end{enumerate}

\newpage

\section{Example}

We have provided an example demonstrating the usage of the ECU simulator in the file located at \plog{test_ecu_simulator.py}
\\

\begin{pythoncode}
   if __name__ == "__main__":
      # Create and start instances of different ECUs using the factory pattern and abstract class
      factory = ECUFactory()

      positive_ecu = factory.create_ecu(ECUType.POSITIVE_ECU, POSITIVE_ECU_IP, POSITIVE_TCP_PORT, POSITIVE_UDP_PORT)
      negative_ecu = factory.create_ecu(ECUType.NEGATIVE_ECU, NEGATIVE_ECU_IP, NEGATIVE_TCP_PORT, NEGATIVE_UDP_PORT)
      # Start positive and negative ECUs
      positive_ecu.start()
      negative_ecu.start()
\end{pythoncode}

In the given example, an instance of the ECU is created in \plog{ecu_simulator.py} by specifying the ECU's IP address, TCP port, and UDP port.
Subsequently, the start method is invoked to initiate its operation.

\textbf{Output:}
\begin{pythoncode}
TCP Server 172.17.0.5 listening on port 13400
UDP Server 172.17.0.5 listening on port 13400
TCP Server 172.17.0.5 listening on port 12346
UDP Server 172.17.0.5 listening on port 12347
\end{pythoncode}

Now you can execute the test by running the file located at \plog{test_ecu_simulator.py}
\\

\begin{pythoncode}
   def test_positive_ecu_simulator():
      try:
         ip = '172.17.0.5'
         ecu_logical_address = 57344

         # Create a DoIPClient instance for positive ECU simulator
         doip = DoIPClient(ip, ecu_logical_address, activation_type=None)

         # Test various interactions
         print(doip.request_diagnostic_power_mode())
         print(doip.request_entity_status())
         print(doip.request_alive_check())
         print(doip.request_activation(1))
         print(doip.get_entity())
         print(doip.request_vehicle_identification(vin="1" * 17))
         print(doip.request_vehicle_identification(eid=b"1" * 6))

      except Exception as e:
         print(f"Error during positive ECU simulation: {e}")
\end{pythoncode}

\newpage

\textbf{Output:}
\begin{pythoncode}
   # Diagnostic power mode response
   DiagnosticPowerModeResponse (0x4004): { diagnostic_power_mode : DiagnosticPowerMode.Ready }
   
   # Entity status response
   EntityStatusResponse (0x4002): { node_type : 1, max_concurrent_sockets : 16, currently_open_sockets : 1, max_data_size : None }
   
   # Alive check response
   AliveCheckResponse (0x8): { source_address : 3584 }
   
   # Routing activation response
   RoutingActivationResponse (0x6): { client_logical_address : 3584, logical_address : 55, response_code : ResponseCode.Success, reserved : 0, vm_specific : None }  
   
   # Get entity response
   (('172.17.0.5', 13400), VehicleIdentificationResponse(b'19676527011956855', 3584, b'11111\x00', b'222222', 0, 0))
   
   # Vehicle identification response
   VehicleIdentificationResponse (0x4): { vin: "19676527011956855", logical_address : 3584, eid : b'11111\x00', gid : b'222222', further_action_required : FurtherActionCodes.NoFurtherActionRequired, vin_sync_status : SynchronizationStatusCodes.Synchronized }
   VehicleIdentificationResponse (0x4): { vin: "19676527011956855", logical_address : 3584, eid : b'11111\x00', gid : b'222222', further_action_required : FurtherActionCodes.NoFurtherActionRequired, vin_sync_status : SynchronizationStatusCodes.Synchronized }
\end{pythoncode}
